\documentclass[a4paper,twoside,12pt]{article}
%
\usepackage{amsmath}
\usepackage[latin9]{inputenc}
\usepackage{exercise}
\usepackage[english]{babel}
\usepackage{tcolorbox}
\usepackage{subcaption}
\usepackage{hyperref}
\usepackage{multirow}
%
\usepackage{fancyhdr}
\pagestyle{fancy}
\usepackage{graphicx}
\fancyhead{} % clear all header fields
\fancyhead[CE,CO] {\bfseries Switches controller}
\fancyfoot{} % clear all footer fields
\fancyfoot[LE,LO]{\bfseries ties.kluter@epfl.ch}
\fancyfoot[CE,CO]{\thepage}
\fancyfoot[RO,RE]{\bfseries CS-473}
\renewcommand{\headrulewidth}{0.4pt}
\renewcommand{\footrulewidth}{0.4pt}
\newtcolorbox{important}{width=\textwidth,colback=red!55,colframe=red!95,title=WICHTIG:}
%
\begin{document}
The buttons, joystick, and dip-switches can either be used in polling-mode or in interrupt-based mode. The base-address of this interface is \texttt{0x50000080}. The different functions are shown below:
\begin{center}
\begin{tabular}{|c|c|l|l|}
\hline
\textbf{byte}&\textbf{word}&\multicolumn{2}{|c|}{\textbf{Functionality:}}\\
\cline{3-4}
\textbf{index}&\textbf{index}&\textbf{read:}&\textbf{write:}\\
\hline
\hline
\texttt{0}&\texttt{0}&Read the dip-switches state&No operation\\
\hline
\multirow{2}{*}{\texttt{4}}&\multirow{2}{*}{\texttt{1}}&Read and clear pressed&Write dip-switch pressed\\
&&dip-switch IRQ generators&IRQ enable mask\\
\hline
\multirow{2}{*}{\texttt{8}}&\multirow{2}{*}{\texttt{2}}&Read and clear released&Write dip-switch released\\
&&dip-switch IRQ generators&IRQ enable mask\\
\hline
\multirow{2}{*}{\texttt{12}}&\multirow{2}{*}{\texttt{3}}&Read the joystick and&\multirow{2}{*}{No operation}\\
&&buttons state&\\
\hline
\multirow{3}{*}{\texttt{16}}&\multirow{3}{*}{\texttt{4}}&Read and clear pressed&Write joystick and\\
&&joystick and buttons&buttons pressed\\
&&IRQ generators&IRQ enable mask\\
\hline
\multirow{3}{*}{\texttt{20}}&\multirow{3}{*}{\texttt{5}}&Read and clear released&Write joystick and\\
&&joystick and buttons&buttons released\\
&&IRQ generators&IRQ enable mask\\
\hline
\multirow{2}{*}{\texttt{24}}&\multirow{2}{*}{\texttt{6}}&Read IRQ latency&No operation\\
&&value&\\
\hline
\multirow{2}{*}{\texttt{28}}&\multirow{2}{*}{\texttt{7}}&Clear all IRQ&Clear all IRQ\\
&&generators&enable masks\\
\hline
\end{tabular}
\end{center}
When addressing the dip-switches, the bit 0 of the 16-bit word corresponds to the dip-switch to the right (nr. 8) of the board. Bit 15 of the 16-bit word corresponds to the dip-switch to the left (nr. 1) of the board in case of the GECKO4Education. In case of the GECKO5Education only the lower 8 bits are available, all others read 0. This coding also holds for the IRQ-masks and IRQ-generators. When addressing the buttons, the GECKO4Education and GECKO5Education have different mappings:
\begin{itemize}
\item \textbf{GECKO4Education} : the button SW2 (see also the \href{https://gecko-wiki.ti.bfh.ch/gecko4education_epfl:buttons}{gecko-wiki}) is mapped on the LSB, and SW7 is mapped on bit 5.
\item \textbf{GECKO5Education} : the button SW1 is mapped on bit 5 and SW5 on bit 9. The joystick is mapped as shown below:
\begin{center}
\begin{tabular}{|c|l|c|l|}
\hline
\textbf{bit nr.}&\textbf{Joystick function:}&\textbf{bit nr.}&\textbf{Joystick function:}\\
\hline
\hline
3&North&2&East\\
\hline
1&South&0&West\\
\hline
4&Center button&-&-\\
\hline
\end{tabular}
\end{center}
\end{itemize}
These coding also holds for the IRQ-masks and IRQ-generators.
\section{Scanning frequency}
The buttons, joystick, and dip-switches are scanned to avoid any dender effects. The standard frequency of this scanning is 1kHz.
\section{IRQ handling}
Interrupts can be enabled for both pressing a button or releasing a button. The interrupts are enabled in the pressed or released IRQ enable masks. In case an interrupt is enabled and a corresponding action is detected (press and/or release) an interrupt is generated. The dip-switches will activate \texttt{IRQ bit2}, the joystick, and buttons will activate \texttt{IRQ bit3} of the user-CPU. To see to sources of the IRQ, the pressed and/or released IRQ generators registers can be read out. Reading these registers will then automatically clear the IRQ-event. If a user want to clear all sources that generated an IRQ, a read of the \texttt{Clear all IRQ generators} register can be performed. To clear all IRQ-enable masks a write to the \texttt{Clear all IRQ enable masks} register can be performed.\\
Finally, the IRQ's generated by the module are active-high and stay active up to the moment the CPU clears the IRQ-generators by the means described above.
\section{IRQ latency counter}
The module also implements a 32-bit latency counter. This counter will count the number of system clock cycles between the moment an IRQ is generated and it is cleared by the CPU.
\end{document}
