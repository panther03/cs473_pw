\documentclass[a4paper,twoside,12pt]{article}
%
\usepackage{amsmath}
\usepackage[latin9]{inputenc}
\usepackage{exercise}
\usepackage[english]{babel}
\usepackage{tcolorbox}
\usepackage{subcaption}
\usepackage{hyperref}
\usepackage{multirow}
%
\usepackage{fancyhdr}
\pagestyle{fancy}
\usepackage{graphicx}
\fancyhead{} % clear all header fields
\fancyhead[CE,CO] {\bfseries LED array controller}
\fancyfoot{} % clear all footer fields
\fancyfoot[LE,LO]{\bfseries ties.kluter@epfl.ch}
\fancyfoot[CE,CO]{\thepage}
\fancyfoot[RO,RE]{\bfseries CS-473}
\renewcommand{\headrulewidth}{0.4pt}
\renewcommand{\footrulewidth}{0.4pt}
\newtcolorbox{important}{width=\textwidth,colback=red!55,colframe=red!95,title=WICHTIG:}
%
\begin{document}
There are two modes in which the LED-Array can be accessed, namely line-based and pixel based. Each of the below section will describe these access modi. Note that:
\begin{itemize}
\item On the GECKO5Education there is an RGB-array that uses a scanning frequency of 1kHz.
\item On the GECKO4Education there is only a single color array, writing to any of the red, green or blue bits will light up the led.
\item The intensity of the LED's cannot be controlled (they are either on or off).
\item This module supports burst mode, or single word mode. Short or byte access will result in a bus-error.
\end{itemize}
\section{Line-based control}
The base address of the line-based control of the LED-array is \texttt{0x50000800}. The offset inside this control mode is 10-bit (byte based) or 8-bit (word based). The functionality of these offset bits is shown below:
\begin{center}
\begin{tabular}{ccc}
\textbf{Byte-based offset:}&&\textbf{Word-based offset:}\\
\begin{tabular}{|c|c|c|c|c|c|c|c|c|c|}
\hline
\textbf{9}&\textbf{8}&\textbf{7}&\textbf{6}&\textbf{5}&\textbf{4}&\textbf{3}&\textbf{2}&\textbf{1}&\textbf{0}\\
\hline
\hline
\multicolumn{2}{|c}{access}&\multicolumn{4}{|c}{line}&\multicolumn{2}{|c|}{color}&\multirow{2}{*}{0}&\multirow{2}{*}{0}\\
\multicolumn{2}{|c}{mode}&\multicolumn{4}{|c}{index}&\multicolumn{2}{|c|}{mode}&&\\
\hline
\end{tabular}&
\hspace*{0.5cm}&
\begin{tabular}{|c|c|c|c|c|c|c|c|}
\hline
\textbf{7}&\textbf{6}&\textbf{5}&\textbf{4}&\textbf{3}&\textbf{2}&\textbf{1}&\textbf{0}\\
\hline
\hline
\multicolumn{2}{|c}{access}&\multicolumn{4}{|c}{line}&\multicolumn{2}{|c|}{color}\\
\multicolumn{2}{|c}{mode}&\multicolumn{4}{|c}{index}&\multicolumn{2}{|c|}{mode}\\
\hline
\end{tabular}\\
\end{tabular}
\end{center}
The different items are:
\begin{itemize}
\item \textbf{Color mode:} The color mode defines which of the LED's are activated in the selected line of LED's. Their definition is:
\begin{center}
\begin{tabular}{|c|l|}
\hline
\textbf{Value:}&\textbf{Mode:}\\
\hline
\hline
\texttt{0}&Write to the blue LED's of the line.\\
\hline
\texttt{1}&Write to the green LED's of the line.\\
\hline
\texttt{2}&Write to the red LED's of the line.\\
\hline
\texttt{3}&Write to the red,green, and blue LED's of the line (white color).\\
\hline
\end{tabular}
\end{center}
\item \textbf{Line index:} This selects the line of the LED array, where line 0 is the top-most line of the array and line 8/9 the bottom line of the array. Note that when line-index is bigger than 9 this will not do anything.
\item \textbf{Access mode:} To set or clear a LED, a word is written. In this word the LSB (bit 0) references the right-most LED in the line, and bit 10 the left-most LED in the line. The bits 31..11 of this word are don't care. How the LED's react on a 1 or 0 is determined by the access mode:
\begin{center}
\begin{tabular}{|c|l|}
\hline
\textbf{Value:}&\textbf{Access mode:}\\
\hline
\hline
\multirow{2}{*}{\texttt{0}}&The bit will be directly written hence a 1 lights up the LED,\\
&a 0 will turn it off.\\
\hline
\multirow{2}{*}{\texttt{1}}&The bit will be or-masked written hence a 1 lights up the LED,\\
&a 0 will leave it as is.\\
\hline
\multirow{2}{*}{\texttt{2}}&The bit will be and-masked written hence a 0 turns off the LED,\\
&a 1 will leave it as is.\\
\hline
\multirow{2}{*}{\texttt{3}}&The bit will be xor-masked written hence a 1 inverts the LED,\\
& a 0 will leave it as is.\\
\hline
\end{tabular}
\end{center}
\end{itemize}
Reading from any of the locations inside this address space will show the current state of the LED's, where the access-mode is ignored.
\section{Pixel-based control}
The RGB-array can also be addressed pixel by pixel. The base-address for this mode is \texttt{0x50000C00}. The offset inside this control mode is 9-bit (byte based) or 7-bit (word based), where the offset indicated the pixel inside the array. Offset 0 is the pixel on the left-top and offset 107/119 (short based) is the pixel on the right-bottom.\\
To each pixel the RGB-value can be written where the red-value is bit 2 of the data-word. The green value is bit 1 of the data-word, and bit 0 is the blue value. Reading from a location will return the current RGB-value of the pixel (LED).
\section{Writing the base address}
\textbf{Important:} Do not use this function if you do not know what you are doing, as it could result in funny results.\\
By default the led controller is at the memory address \texttt{0x50000800}, however, it can be moved in the memory space by writing a complete 32-bit memory address. Please note
that the bits 9..0 of this address must be 0. The base-address register can be accessed by a word offset of \texttt{0x1FF} from the current base address. Hence in the start-up mode on address \texttt{0x50000FFC}.

\end{document}
